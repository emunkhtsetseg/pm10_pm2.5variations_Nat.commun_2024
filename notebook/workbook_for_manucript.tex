% Options for packages loaded elsewhere
\PassOptionsToPackage{unicode}{hyperref}
\PassOptionsToPackage{hyphens}{url}
\documentclass[
  11pt,
]{article}
\usepackage{xcolor}
\usepackage[margin=1.0in]{geometry}
\usepackage{amsmath,amssymb}
\setcounter{secnumdepth}{-\maxdimen} % remove section numbering
\usepackage{iftex}
\ifPDFTeX
  \usepackage[T1]{fontenc}
  \usepackage[utf8]{inputenc}
  \usepackage{textcomp} % provide euro and other symbols
\else % if luatex or xetex
  \usepackage{unicode-math} % this also loads fontspec
  \defaultfontfeatures{Scale=MatchLowercase}
  \defaultfontfeatures[\rmfamily]{Ligatures=TeX,Scale=1}
\fi
\usepackage{lmodern}
\ifPDFTeX\else
  % xetex/luatex font selection
\fi
% Use upquote if available, for straight quotes in verbatim environments
\IfFileExists{upquote.sty}{\usepackage{upquote}}{}
\IfFileExists{microtype.sty}{% use microtype if available
  \usepackage[]{microtype}
  \UseMicrotypeSet[protrusion]{basicmath} % disable protrusion for tt fonts
}{}
\makeatletter
\@ifundefined{KOMAClassName}{% if non-KOMA class
  \IfFileExists{parskip.sty}{%
    \usepackage{parskip}
  }{% else
    \setlength{\parindent}{0pt}
    \setlength{\parskip}{6pt plus 2pt minus 1pt}}
}{% if KOMA class
  \KOMAoptions{parskip=half}}
\makeatother
\usepackage{graphicx}
\makeatletter
\newsavebox\pandoc@box
\newcommand*\pandocbounded[1]{% scales image to fit in text height/width
  \sbox\pandoc@box{#1}%
  \Gscale@div\@tempa{\textheight}{\dimexpr\ht\pandoc@box+\dp\pandoc@box\relax}%
  \Gscale@div\@tempb{\linewidth}{\wd\pandoc@box}%
  \ifdim\@tempb\p@<\@tempa\p@\let\@tempa\@tempb\fi% select the smaller of both
  \ifdim\@tempa\p@<\p@\scalebox{\@tempa}{\usebox\pandoc@box}%
  \else\usebox{\pandoc@box}%
  \fi%
}
% Set default figure placement to htbp
\def\fps@figure{htbp}
\makeatother
% definitions for citeproc citations
\NewDocumentCommand\citeproctext{}{}
\NewDocumentCommand\citeproc{mm}{%
  \begingroup\def\citeproctext{#2}\cite{#1}\endgroup}
\makeatletter
 % allow citations to break across lines
 \let\@cite@ofmt\@firstofone
 % avoid brackets around text for \cite:
 \def\@biblabel#1{}
 \def\@cite#1#2{{#1\if@tempswa , #2\fi}}
\makeatother
\newlength{\cslhangindent}
\setlength{\cslhangindent}{1.5em}
\newlength{\csllabelwidth}
\setlength{\csllabelwidth}{3em}
\newenvironment{CSLReferences}[2] % #1 hanging-indent, #2 entry-spacing
 {\begin{list}{}{%
  \setlength{\itemindent}{0pt}
  \setlength{\leftmargin}{0pt}
  \setlength{\parsep}{0pt}
  % turn on hanging indent if param 1 is 1
  \ifodd #1
   \setlength{\leftmargin}{\cslhangindent}
   \setlength{\itemindent}{-1\cslhangindent}
  \fi
  % set entry spacing
  \setlength{\itemsep}{#2\baselineskip}}}
 {\end{list}}
\usepackage{calc}
\newcommand{\CSLBlock}[1]{\hfill\break\parbox[t]{\linewidth}{\strut\ignorespaces#1\strut}}
\newcommand{\CSLLeftMargin}[1]{\parbox[t]{\csllabelwidth}{\strut#1\strut}}
\newcommand{\CSLRightInline}[1]{\parbox[t]{\linewidth - \csllabelwidth}{\strut#1\strut}}
\newcommand{\CSLIndent}[1]{\hspace{\cslhangindent}#1}
\setlength{\emergencystretch}{3em} % prevent overfull lines
\providecommand{\tightlist}{%
  \setlength{\itemsep}{0pt}\setlength{\parskip}{0pt}}
\usepackage{helvet} % Helvetica font
\renewcommand*\familydefault{\sfdefault} % Use the sans serif version of the font
\usepackage[T1]{fontenc}

\usepackage[none]{hyphenat}

\usepackage{setspace}
\doublespacing
\setlength{\parskip}{1em}

\usepackage{lineno}

\usepackage{pdfpages}
\usepackage{bookmark}
\IfFileExists{xurl.sty}{\usepackage{xurl}}{} % add URL line breaks if available
\urlstyle{same}
\hypersetup{
  pdftitle={Spatio-temporal distinct patterns in variations of PM\_\{10\} and PM\_\{2.5\} relative to the recent drivings of emission sources in Mongolia},
  hidelinks,
  pdfcreator={LaTeX via pandoc}}

\title{\textbf{Spatio-temporal distinct patterns in variations of
\(PM_{10}\) and \(PM_{2.5}\) relative to the recent drivings of emission
sources in Mongolia}}
\author{}
\date{\vspace{-2.5em}}

\begin{document}
\maketitle

\vspace{35mm}

Running title: INSERT RUNNING TITLE HERE

\vspace{35mm}

Munkh\^{}1, Joeseph P. Schmo\^{}2, Sally J. Rivers\^{}1, Patrick D.
Schloss\textsuperscript{1}\(\dagger\)

\vspace{40mm}

\(\dagger\) To whom correspondence should be addressed:
\href{mailto:pschloss@umich.edu}{\nolinkurl{pschloss@umich.edu}}

1. Department of Microbiology and Immunology, University of Michigan,
Ann Arbor, MI 48109

2. Other department contact information

\newpage
\linenumbers

\subsection{Abstract}\label{abstract}

Storyline:

\begin{enumerate}
\def\labelenumi{\arabic{enumi}.}
\tightlist
\item
  A new pattern is emerged
\item
  Air quality in urban sites is episodically dictated by dust events in
  spring or late autumn, yet seasonally governed by anthropogenic
  emissions in winter. {[}Air quality is governed by natural dust
  emission, and anthropogenic emissions{]}
\item
  With recent growing interest in urban life style, and combustion of
  coal/oyutolgoi for heating winter conditions results a highly increase
  in not only capital city but also towns
\item
  In a result, spring coarse dust, plus winter fine pollutants
\item
  spring coarse dust is immediately transported and deposited in the
  source area, whereas winter fine pollutants is permanently stayed in
  the source area due to stagnant atmosphere govern over entire
  country., perhaps floating in the near surface, deposits in the
  surface{]}
\item
  Alarms, the Mongolian dust in the spring, optical properties might be
  shifted; this gives \ldots{} Gobi dust and sand storms has become
  tuiren, from the shoroon shuurga. which clearly requires the
  attention.
\item
  r ratio shows \ldots{} emission source; dust might carry anthropogenic
  fine particulates as well. \newpage
\end{enumerate}

\subsection{Introduction}\label{introduction}

\begin{itemize}
\item
  More importantly, the increased concentrations of particulate matters
  has a significant effects on the climate system, altering the solar
  incidence, cloud formation, and precipitation. Because a comprehensive
  research studies on dust and aerosol, particularly from the dust
  source regions is invaluable.On the other hand, concentrations of
  particulate matter is ephederemal, yet vary depending on whether the
  pollution cause is natural or industrial, local or transported,
  seasonal or non-seasonal.
\item
  It is well-informed that concentrations of air particulate matter
  solely depend on urbanization and economic situations to the area of
  the interest of the country. Globally, 7.3 billion people are directly
  exposed to unsafe average annual PM2.5 concentrations, and 80\% of
  them living in low- and middle-income countries, where economies often
  rely heavily on polluting industries. A similar pattern of the
  significant disparities in air quality among income and racial/ethnic
  groups, as well as between urban and rural areas was reported in USA
  (Liu et al., 2021). Despite this disparity, meteorological effects
  such as dust storm, stagnant weather plays important role in the
  spatiotemporal variability of PM10 and PM2.5. For an instance, in
  Mongolia, the atmospheric environment related to urban and rural air
  pollution are strongly characterized by its temperate and dry climatic
  conditions. Siberian anticyclonic activity governed over Mongolia,
  which create a significant vulnerability to winter air pollution in
  the populated areas. The monthly mean concentrations of PM10 (PM2.5)
  reached annual maximum in December and January due to winter synoptic
  governing conditions in Ulaanbaatar, capital city of Mongolia
  (Jugder). Despite this, the spring dust storms creates another
  polluted season in UB. On spring, the dust storm from the Gobi Desert
  contribute significantly to increased aerosols in the atmosphere and
  ambient air pollution, leading to sporadic peaks in PM10
  concentrations reaching as high as 64-234 \(\mu g m^{-3}\) per day or
  exceeding 6000 \(\mu g m^{-3}\) per hour (Jugder). A such changes in
  PM10 and PM2.5 to stagnant weather conditions, and local or
  transported dust was also observed in other countries China (Wang),
  Korea (Kim) and Japan (). Many research findings/Numerous research
  findings have advanced the field, and air quality indices is widely
  used for providing guidance, and public perception of air quality has
  been improved (Mirabelli et al., 2020).
\item
  Demonstrating temporal and spatial variations of air particulate
  matter has become important for understanding characteristics of
  particulate matter in the climate system, providing valuable
  information for well-established air quality measures, and
  illustrating the good trace data for health studies. Because
  particulate pollutants have a great impact on human health (Dockery
  and Pope,1994; Harrison and Yin, 2000; Hong et al., 2002), high
  atmospheric concentrations of these pollutants was a major concern
  particularly in urban areas, in the last 2-3 decades. Recent studies
  highlight that even low concentrations of these pollutants can lead to
  various health issues, and may associate with morbidity and mortality
  across the life span (Zigler et al., 2017). Children exposed to high
  levels of air pollution show increased rates of asthma, decreased lung
  function growth, and increased risk of early markers of cardiovascular
  disease (Bourdrel et al., 2017; Gauderman et al., 2015; Hehua et al.,
  2017). Short-term exposure with high level of PM10 resulted the
  chronic cardiovascular disease in Mongolia (Enkhjargal 2020). In
  addition to these health issues, (prenatal) neurodevelopmental impacts
  such as effects on intelligence, attention, autism, and mood, while
  aging populations experience accelerated cognitive decline when
  exposed to high levels of pollution is detected (Power et al., 2016).
  Long-term exposure to low levels of particulate matter, such as
  concentrations as low as 10 \(\mu g m^{-3}\) (equilibrium to WHO Air
  Quality Guidelines), has been linked to increased lung cancer in the
  EU (Hvidtfeldt et al.~2021), with similar evidences reported in Canada
  (Bai et al., 2019), and significantly higher rates captured in China
  with concentrations up to 30 \(\mu g m^{-3}\). Apparently, pollutants
  of particulate matters has effects to various health issues with the
  different thresholds and exposure durations. However, more in-depth
  and diversified research on air pollution and its health effects is
  essential, with the detailed information is necessary (Tan et al 2021)
  to have accuracy of assessing exposure to air pollution during
  developmentally relevant time periods, such as trimesters or months
  (Becerra et al., 2013; Gong et al., 2014; Kalkbrenner et al., 2014) or
  weeks (Chiu et al., 2016)
\item
  Therefore, we aimed to demonstrate the distinct temporal and spatial
  variations of PM2.5 and PM10 across urban and rural Mongolia using
  extensive data from 2008 to 2020. The present study will contribute
  significantly to the understanding of air particulate matter patterns
  in Mongolia and providing comprehensive data insights for policymakers
  and public health sectors. Our findings is useful not only for
  addressing national health impacts but also beneficial for
  understanding air particulate matter as ambient air pollution, and
  tackling atmospheric aerosol effects in the climate system, and
  revealing their transboundary effects to the downwind regions in
  South-east Asia. \newpage
\end{itemize}

\subsection{Results and Discussion}\label{results-and-discussion}

\textsuperscript{1} dksl ss\textsuperscript{2} \textbf{Figure 3.
Geographic locations of study sites} is shown in the wind speed map and
elevation maps.

\newpage

\subsection{Conclusions}\label{conclusions}

\newpage

\subsection{Materials and Methods}\label{materials-and-methods}

\subsubsection{A description of study
sites}\label{a-description-of-study-sites}

According to the spatial magnitude of wind stress in Mongolia (Figure
1), the largest magnitude of wind speed is on the Gobi sites,
particularly those located in the southeast edge of the country.

\begin{itemize}
\tightlist
\item
  The impact of high winds on plant diversity varies across
  environmental gradients of precipitation and soil fertility (Milchunas
  et al., 1988).
\item
  In the desert steppe zone, species richness was lower in the drier
  years but did not vary with grazing pressure.
\item
  In the steppe zone, species richness varied significantly with grazing
  pressure but did not vary between years. Species richness is not
  impacted by grazing gradient in desert steppe, but it is in the steppe
  (Cheng et al., 2011).
\end{itemize}

In the last 2 decades, due to poverty and natural disasters there is
population immigration has taken place from the rural to urban,
especially to capital city of Mongolia. Due to tiny infrastructure to
provide the mega city with the dense population, it introduces the urban
pollution. Therefore, Ulaanbaatar air particulate matter mainly reflects
the coal burning, and partly, natural dust.

Consequently, the atmospheric environment and climate for Mongolian Gobi
has been impacted the most by frequent dust and and sand storm in the
spring.

Our study was carried out in Dalanzadgad (town center) (Tbl. 1; 43.57°N,
104.42°E), Sainshand (Tbl. 1; 44.87°N, 110.12°E) and Zamyn-Uud (Tbl. 1;
43.72°N, 111.90°E) in the Gobi Desert, and at Ulaanbaatar (Tbl.??.??°N,
104.42°E) (city center) located in the temperate Mongolian steppe of
Mongolia (Figure 2). Nomads and settlements of this sum have raised a
large number of livestock, and they rank at number 30 out of 329 sums
for the largest number of livestock raised per sum (Saizen et al.,
2010). In the last decade, the number of dust events associated with
wind erodibility increased by 30 \% in Bayan-Önjüül (Kurosaki et al.,
2011). This is an area where dust emissions activity has been monitored
on a long-term basis (Shinoda et al., 2010a) at a dust observation site
(DOS) adjacent to the study site (Fig. 1a). According to long-term
meteorological observations made at the monitoring station of the
Institute of Meteorology and Hydrology of Mongolia located near the
site, the prevailing wind direction is northwest. Mean annual
precipitation is 163 mm, and mean temperature is 0.1◦C for the period
1995 to 2005 (Shinoda et al., 2010b). Soil texture is dominated by sand
(98.1 \%, with only 1.3 \% clay and 0.6 \% silt; Table 1; Shinoda et
al., 2010a). Insert figure legends with the first sentence in bold, for
example:

\textbf{Figure 1. Geographic locations of study sites} is shown in the
wind speed map and elevation maps.

\textbf{Table 1. A description of datasets obtained at the sites}

\textbf{Figure 2.} is shown in the wind speed map and elevation maps.

\cite{Schloss2009} \newpage

\subsection{References}\label{references}

This is an \href{http://rmarkdown.rstudio.com}{R Markdown} Notebook.
When you execute code within the notebook, the results appear beneath
the code.

Try executing this chunk by clicking the \emph{Run} button within the
chunk or by placing your cursor inside it and pressing
\emph{Cmd+Shift+Enter}.

\pandocbounded{\includegraphics[keepaspectratio]{workbook_for_manucript_files/figure-latex/unnamed-chunk-1-1.pdf}}

Add a new chunk by clicking the \emph{Insert Chunk} button on the
toolbar or by pressing \emph{Cmd+Option+I}.

When you save the notebook, an HTML file containing the code and output
will be saved alongside it (click the \emph{Preview} button or press
\emph{Cmd+Shift+K} to preview the HTML file).

The preview shows you a rendered HTML copy of the contents of the
editor. Consequently, unlike \emph{Knit}, \emph{Preview} does not run
any R code chunks. Instead, the output of the chunk when it was last run
in the editor is displayed.

\phantomsection\label{refs}
\begin{CSLReferences}{0}{0}
\bibitem[\citeproctext]{ref-Schloss2009}
\CSLLeftMargin{1. }%
\CSLRightInline{Schloss, P. D. \emph{et al.}
\href{https://doi.org/10.1128/aem.01541-09}{Introducing mothur:
Open-source, platform-independent, community-supported software for
describing and comparing microbial communities}. \emph{Applied and
Environmental Microbiology} \textbf{75}, 7537--7541 (2009).}

\bibitem[\citeproctext]{ref-Munkh2017}
\CSLLeftMargin{2. }%
\CSLRightInline{Munkhtsetseg, E. \emph{et al.}
\href{https://doi.org/10.5194/acp-17-11389-2017}{Anthropogenic dust
emissions due to livestock trampling in a mongolian temperate
grassland}. \emph{Atmospheric Chemistry and Physics} \textbf{17},
11389--11401 (2017).}

\end{CSLReferences}

\end{document}
